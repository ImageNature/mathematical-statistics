% !TeX TS-program = xelatex
\documentclass[a5paper,12pt]{article}
\usepackage{my-windows-math}
\title{数理统计}
\author{ImageNature}
\date{\today}
\begin{document}
\maketitle


抛开实际背景,总体就是一堆有大有小的数,因此用一个概率分布去描述和归纳总体是恰当的.
某种意义上来说,总体就是一个分布,它的数量指标就是服从分布的随机变量.
因此从总体中抽样和从某分布中抽样是同一个意思.
\section{绪论}

\begin{enumerate}
  \item 若将样本观测值有小到大进行排列,记作 $X_{(1)},\cdots,X_{(n)}$ ,则称为有序样本,且可以用有序样本定义经验分布函数. 有序样本对应的是次序统计量.
  \item 统计量是一类函数,统计量的分布称为抽样分布. 尽管统计量不依赖于未知参数,但是它的分布是依赖于未知参数的.
  \item 设 $x_1,\cdots,x_n$ 是来自某个总体的样本,$\overline{x}$ 是样本均值
  \begin{itemize}
    \item 若总体分布是 $N(\mu,\sigma^2)$,则 $\overline{x}$ 的精确分布为 $N(\mu,\sigma^2/n)$
    \item 若总体分布不是正态分布或未知,$E(x) = \mu,Var(x) = \sigma^2$ 存在,
    则当 $n$ 较大时 $\overline{x}$ 的极限分布(渐进分布) 为 $N(\mu,\sigma^2/n)$ .这里渐进分布是 $n$ 较大时的近似分布.
  \end{itemize}
 \item 样本方差时度量样本散布大小的统计量,样本方差定义为
 \[
 s_n^2 = \frac1 n \sum_{i=1}^{n} (x_i - \overline{x})^2
 \]
 为了更方便地构造无偏统计量,一般会定义为
 \[
 s^2 = \frac{1}{n-1}\sum_{i=1}^{n}(x_i - \overline{x})^2
 \]
  注意不同定义下的样本方差的表示($s$ 有没有 下指标 $n$) ,  $s^2$ 更加常用. 在样本方差的定义中, $n$ 为样本量, $n-1$ 称为偏差平方和的自由度. 自由度的含义是:
 在 $\overline{x}$ 确定后, $n$ 个偏差 $x_1 - \overline{x},\cdots,x_n - \overline{x}$ 只有
 $n-1$ 个偏差可以自由变动,因为其和为 $0$.




  \item 样本偏度和样本峰度都是中心矩的函数,如果数据完全对称,那么样本偏度就是 $0$.
  \begin{itemize}
    \item 样本偏度大于 $0$ ,表示样本的右尾长,即样本中有几个很大的数.
    \item 样本偏度小于 $0$ ,表示样本的左尾长,即样本中有几个很小的数.
    \item 样本峰度大于 $0$ ,分布曲线在峰值附近比正态分布更陡峭,尾部更细——尖顶型.
    \item 样本峰度小于 $0$ , 分布曲线在峰值附近比正态分布更平坦,尾部更粗——平顶型.
  \end{itemize}
  \item 在同一样本中,$x_1,\cdots,x_n$ 是独立同分布的, 但是次序统计量 $x_{(1)},\cdots,x_{(n)}$ 并不独立,分布也不相同.
  \item 设总体密度函数为 $p(x)$, $x_p$ 为其 $p$ 分位数, $p(x)$ 在 $x_p$
  处连续且  $p(x_p)>0$,那么当  $n\to \infty$ 时,样本的 $p$ 分位数 $m_p$
  的渐进分布为
  \[
  m_p \sim N\left( x_p, \frac{p(1-p)}{np^2(x_p)}\right)
  \]
\end{enumerate}



\section{三大抽样分布}
许多统计推断是基于正态分布的假设, 因此有必要了解以标准正态分布为基础构造的三个常用的分布.

\subsection{$\mathcal{X}^2$ 方分布}
设 $X_1,\cdots,X_n$ i.i.d. 于标准正态分布  $N(0,1)$, 则有自由度为 $n$ 的 $\mathcal{X}^2$ 分布
\[
\mathcal{X}^2 = \mathcal{X}_1^2+ \cdots +\mathcal{X}_n^2
\]
记为 $\mathcal{X}^2 \sim \mathcal{X}(n)$.

\begin{enumerate}
  \item 卡方分布与伽玛分布的关系
  \[
  \mathcal{X}^2(n) = Ga(\frac{n}{2}), \frac{1}{2}
  \]
  期望为 $n$ ,方差为 $2n$.
  \item 卡方分布一个重要的定理是: 设 $x_1,\cdots,x_n$ 来自正态分布 $N(\mu,\sigma^2)$ 的样本,其样本均值和样本方差分别为
  \[
  \begin{aligned}
    \overline{x} &= \frac1n \sum_{i=1}^{n} x_i \\
    s^2 &= \frac{1}{n-1} \sum_{i=1}^{n} (x_i - \overline{x})^2
  \end{aligned}
  \]
  则有
  \begin{itemize}
    \item $\overline{x}$ 与  $s^2$ 相互独立.
    \item $\overline{x} \sim N(\mu,\sigma^2/n)$.
    \item $\frac{(n-1)s^2}{\sigma^2} \sim \mathcal{X}^2(n-1)$
  \end{itemize}

  \item $\mathcal{X}^2$ 的定义是$\mathcal{X}^2 = X_1^2+\cdots+X_n^2$ 的分布,
  其中 $X_i$ 是i.i.d.的标准正态分布 $N(0,1)$. 设想一个 $n$  维向量$(X_1,\cdots,X_n)$,从原点到它的长度的平方就是 $\mathcal{X}^2$.
  所以,卡方的物理含义是刻画了一个 $n$  维向量的长度的平方的分布,
  这个向量的每个维度都是按标准正态随机生成的.

  \item 举个机器学习的例子,假设样本特征有n维,并且样本抽取满足一个多元正态分布(事实上每个维度都是独立的),那么这个样本向量的长度平方的期望是 $n$ ,因为 $E(\mathcal{X}^2)=n$.
  \item  再举一个高维球的例子,在 $n$  维里随机抽取这么一个向量,落在半径为 $1$  的高维球里的概率,
  随着 $n$  变大而越来越小,因为 $\mathcal{X}^2$ 的 p.d.f. 整体右移了,$P(\mathcal{X}^2<1)$越来越小

\end{enumerate}


\subsection{ $F$ 分布}
设随机变量 $X_1 \sim \mathcal{X}^2(m), X_2\sim \mathcal{X}^2(n)$, $X_1$ 与  $X_2$ 独立,则有自由度为 $m$ 和 $n$ 的 $F$ 分布
\[
F = \frac{X_1/m}{X_2/n}
\]
记作  $F \sim F(m,n)$ ,其中  $m$ 为分子自由度, $n$ 为分母自由度.

\begin{enumerate}
  \item 推论:设 $x_1,\cdots,x_m$ 是来自 $N(\mu_1,\sigma_1^2)$ 的样本,
  $y_1,\cdots,y_n$ 是来自 $N(\mu_2,\sigma_2^2)$ 的样本,且两样本相互独立
  记
  \[
  s_x^2 = \frac{1}{m-1} \sum_{i=1}^{m}(x_i - \overline{x})^2, \,
  s_y^2 = \frac{1}{n-1} \sum_{i=1}^{n}(x_i - \overline{x})^2,
  \]
  其中
  \[
  \overline{x} = 1/m\sum_{i=1}^{m}x_i, \,
  \overline{y} = 1/n\sum_{i=1}^{n}y_i
  \]
  则有
  \[
  F =  \frac{s_x^2 /\sigma_1^2}{ s_y^2/\sigma_2^2} \sim F(m-1,n-1)
  \]
  特别的,若 $\sigma_1^2 = \sigma_2^2$, 则有
  \[
  F=\frac{s_x^2}{s_y^2} \sim F(m-1,n-1)
  \]
\end{enumerate}


\subsection{ $t$ 分布}
设随机变量  $X_1$ 与  $X_2$ 独立且  $X_1 \sim N(0,1), X_2 \sim \mathcal{X}^2(n)$, 则有自由度为 $n$ 的 $t$ 分布
\[
t = \frac{X_1}{(X_2/N)^{1/2}}
\]
记作  $t \sim t(n)$
\begin{enumerate}
  \item  $t$ 分布的密度函数的图像是一个关于纵轴对称的分布,与标准正态分布的密度函数形状类似,只是它的峰比标准正态分布低一点,尾部的概率比标准正态分布大一些.
  \begin{itemize}
    \item 自由度为 $1$ 的 $t$ 分布就是标准柯西分布,其均值不存在.
    \item  $n>1$ 时, $t$ 分布的数学期望存在且为 $0$.
    \item  $n>2$ 时, $t$ 分布的方差常年在且为 $n/(n-2)$.
    \item  当自由度较大时, $t$ 分布可以用  $N(0,1)$ 近似.
  \end{itemize}
  \item 推论:设  $x_1,\cdots,x_n$ 是来自正态分布  $N(\mu,\sigma^2)$ 的一个样本,$\overline{x}$ 与 $s^2$ 分别是该样本的样本均值和样本方差,则有
  \[
  t = \frac{n^{1/2} (\overline{x} - \mu)}{s} \sim t(n-1)
  \]
\end{enumerate}



\section{充分统计量}
统计上将样本进行加工不损失信息称为充分性. 当给定了一个统计量的值后,
也就知道了样本中关于参数的所有信息,剩下的其它信息就没有什么价值了,
这正是统计量具有充分性的含义.


\end{document}