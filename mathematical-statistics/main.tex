\documentclass[openany]{book}
\usepackage[slide]{slide-article}
%\mcmsetup{titlepage = true}
\title{Seminar-数理统计}
\subtitle{绪论与第一周作业}
\author{姜青娥}
\institution{克莱登大学}
\pretimes{2023年4月26日}

\begin{document}
\maketitle

%\frontmatter
    \tableofcontents

\mainmatter
\chapter{绪论}
%\begin{myitem}
%  \item First item
%\end{myitem}
%\begin{enumerate}
%  \item First.
%\end{enumerate}

\section{基本概念}
\begin{example}
  一批产品共 10000 件,其中有正品有废品,为估计废品率,从中抽取 100 件进行检查.

  10000 件产品为总体,每件样本为个体,100 件为样本,100 叫样本大小,或样本容量.
  这个行为叫抽样.
\end{example}

若总体中的个体的数目为有限个,则称为有限个体. 无限个体的例子比如产品的寿命问题,时间
的取值是无限的.

我们一般关心的是个体上的数量指标,比如寿命,尺寸等. 个体上的数量指标带有随机性,因此
可以把该数量指标看成随机变量(random variable, r.v.)

数量指标在总体上的分布情况就是随机变量的分布.
数量指标就是可以理解为用数字代替的特征. 比如
\begin{equation}\label{1}
  X=\begin{cases}
    1& \text{废品} \\
    0& \text{正品}
  \end{cases}
\end{equation}
中,特定个体上的数量指标是 $r.v. X$ 的观察值.
\begin{definition}
  统计问题研究的对象的全体称为总体. 在数理统计这门课中,总体可以用一个随机变量及其
  概率分布来描述.
\end{definition}
有时候总体可以用 $r.v. X$ 、分布函数 $F$ 来表示,若 $F$ 也有密度 $f$,也可以用密
度函数来表示.

当从一个总体中抽取样本大小为 $n$ 的样本 $X_1,\cdots,X_n$ 时,它们一定是独立同分
布的,记作 {\bfseries i.i.d.} .

当个体上的数量指标不止一项时,则用随机向量来表示. 比如研究某地区的气温和降雨量,则可
以用 $T$ 表示温度,$R$ 表示降雨量, 总体用二维随机变量 $(T,R)$ 或者联合分布
$F(t,r)$ 来表示. 假如 $F$ 有密度 $f$ ,也可以用联合密度 $f(t,r)$ 来表示.

\begin{remark}
  通过上面的表达,我们发觉,分布可能不存在密度.
\end{remark}

\begin{definition}
  样本 $\mathbf{X} = (X_1,\cdots,X_n)$ 可能取值的总体,构成样本空间
  $\mathscr{X}$.
\end{definition}

\begin{example}
  打靶试验,每次三发,考察中靶的环数. 如样本 $\mathbf{X} = (5,1,9)$ 表示反三次
  打靶分别中 5 环,1 环和 9 环. 此时的样本空间为
  \begin{equation}
   \mathscr{X} = \{(x_1,x_2,x_3):x_i = 0,\cdots,10,i = 1,2,3\}
  \end{equation}
\end{example}

\begin{definition}
  设 $X_1,\cdots,X_n$ 为从总体 $F$ 中抽取的容量为 $n$ 的样本,若
  \begin{itemize}
    \item $X_1,\cdots,X_n$ 相互独立
    \item $X_1,\cdots,X_n$ 同分布
  \end{itemize}
  则称 $X_1,\cdots,X_n$ 为简单随机样本.
\end{definition}
有放回的抽样获得的样本就是简单随机样本.

\begin{remark}
  统计模型就是样本分布. 统计上把出现在样本分布中的未知的常数成为参数,多个参数组成
  参数向量. 如正态分布中的 $(a,\sigma)$.
\end{remark}
这些参数需要通过样本去估计. 参数取值的范围称为参数空间. 比如正态分布中的 $\Theta
= \{(a,\sigma):a>0,\sigma>0\}$

\begin{remark}
  样本分布包含未知参数,由于参数的取值不同,因此样本的分布就不止一个,当参数取不同的
  值得到的不同的分布一起构成一个分布族. 记为 $\mathscr{F} =
  \{f(x,\lambda):\lambda >0\} $
\end{remark}
因此更确切地说,统计模型就是样本分布族.


\section{统计量}
由样本算出来的量就叫做统计量. 统计量是一个函数,但是在数理统计中的统计量是指具体的函
数,不能泛指,不能含有未知参数.
\begin{definition}
  设 $\mathbf{X} \sim P_\theta(\theta \in \Theta)$ 是一个统计模型,则定义
  在样本空间上的任何函数 $T(x)(x\in\mathscr{X})$ 都称为统计量.
\end{definition}
\begin{enumerate}
  \item  样本均值
  \begin{equation}
   \overline{X} = \frac{1}{n}\sum_{i=1}^{n}X_i
  \end{equation}
  \item 样本方差
  \begin{equation}
   S^2 = \frac{1}{n-1}\sum_{i=1}^{n}(X_i - \overline{X})^2
  \end{equation}
  $n-1$ 称为自由度.
  \item 次序统计量
  设 $X_1,\cdots,X_n$ 是从总体中抽取的样本,将其按照大小排列为
  $X_{(1)},\cdots,X_{(n)} $.
  按照排列组成的向量$(X_{(1)},\cdots,X_{(n)})$ ,叫做样本的次序统计量.
  \item 样本变异系数
  设 $X_1,\cdots,X_n$ 为从总体 $F$ 中抽取的样本,则称
  \begin{equation}
   \hat{\nu} = \frac{S_n}{\overline{X}}
  \end{equation}
  \item 样本的 $k$ 阶原点矩
  \begin{equation}
   a_{n,k} = \frac{2}{n} \sum_{i=1}^{n} X_i^k,  k \in 1,\cdots,n
  \end{equation}
  \item 样本的 $k$ 阶中心矩
  \begin{equation}
   m_{n,k} = \frac{1}{n} \sum_{i=1}^{n} (X_i - \overline{X})^k,  k
   \in 1,\cdots,n
  \end{equation}
  \item
  \begin{definition}
   设  $X_1,\cdots,X_n$ 为自总体 $F(x)$ 中抽取的 i.i.d. 样本,将其按大小排列
   为  $X_{(1)} \leq \cdots \leq X_{(n)}$,对任意实数 $x$,称下列函数
   \begin{equation}
    F_n(x)\begin{cases}
      0& x\leq X_{(1)} \\
      \frac kn & X_{(k)} < X \leq X_{(k+1)} ,k = 1,\cdots,n-1  \\
      1& X_{(n)} < X
    \end{cases}
   \end{equation}
  为经验分布函数.
  \end{definition}
  经验分布函数是单调、非降,左连续函数. 并且它仅依赖于样本 $X_1,\cdots,X_n$ 的函
  数,因此它是统计量.
\end{enumerate}

\begin{enumerate}
  \item 由中心极限定理,当 $n \rightarrow \infty$ 时有
  \begin{equation}
   \frac{\sqrt{n}(F_n(x) - F(x))}{\sqrt{F(x)(1-F(x))}}
   \xrightarrow{\mathscr{L}} N(0,1)
  \end{equation}
  其中 $\mathscr{L}$ 表示依分布收敛
 \item 由 Bernoulli或(辛钦)大数定律,则当 $n \rightarrow \infty$ 时有
 \begin{equation}
  F_n(x) \xrightarrow{P} F(x)
 \end{equation}
  \item 由强大数定律,则有
  \begin{equation}
   P\left(\lim_{n\rightarrow \infty} F_n(x) =F(x)\right) = 1
  \end{equation}
 \item 更进一步,有格里汶科定理(Glivenko-Cantelli Theorem)
 \begin{definition}
   设 $F(x)$ 为 $r.v. X$ 的分布函数,  $X_1,\cdots,X_n$ 为取自总体 $F(x)$
   的简单随机样本,$F_n(x)$ 为其经验函数, 记 $D_n = \sup\limits_{-\infty
   <x <\infty} |F_n(x) - F(x)|$,则有
   \begin{equation}
    P\left( \lim\limits_{n\rightarrow \infty} D_n = 0\right) = 1
   \end{equation}
 \end{definition}


\end{enumerate}

\section{第一周课后作业参考答案}

\begin{enumerate}
  \item 试举出一个有限总体的例子,并指出其概率分布.~{\color{cyan} (2 分)}

  {\color{red} \heiti 【解】} {\color{teal} \kaishu
    检验一批产品(假设有 $n$ 件)的质量,每一件产品的检验结果为合格或不合格,记录
    检验结果中合格品的数量 $X$,
    则 $X$ 的可能取值为 $0, 1, 2, \ldots, n$,这是一个有限总体的问题。如果记任
    意一件产品为合格品的概率为 $p$,
    则这一批产品当中合格品数量 $X$ 的概率分布为
    $$ P \left( X = k \right) = \binom{n}{k}
    p^k \left( 1 - p \right)^{n-k}~, \quad
    k = 0, 1, 2, \ldots, n $$}

\begin{remark}
  一般来说, 若随机变量 \(X\) 服从参数为 \(n\) 和 \(p\) 的二项分布, 我们记作
  \(X \sim b(n, p)\) 或 \(X \sim B(n, p)\) 。 \(n\) 次试验中正好得到
  \(k\) 次成功的概率由分布函数或概率质量函数给出:
  \[
  f(k, n, p)=\operatorname{Pr}(X=k)=\binom{n}{k} p^k(1-p)^{n-k}
  \]
  对于 \(k=0,1,2, \cdots, n\),
  其中
  \[
  \binom{n}{k}=\frac{n !}{k !(n-k)  !}
  \]
  是二项式系数(这就是二项分布的名称的由来), 又记为 \(C(n, k),{ }_n C_k\),
  或 \({ }^n C_k\) . 该公式可以用以下方法理解: 我们希望有 \(k\)
  次成功(概率为
  \(p^k\) )和 \(n-k\) 次失败(概率为 \((1-p)^{n-k}\) )。然而, \(k\) 次成功
  可以在 \(n\) 次试验的任何地方出现, 而把 \(k\) 次成功分布在 \(n\)
  次试验中共有 \(C(n, k)\) 个不同的方法.
\end{remark}

  \item 试举出一个无限总体的例子,并指出其概率分布.~{\color{cyan} (2 分)}

  {\color{red} \heiti 【解】} {\color{teal} \kaishu
    检验一批产品的寿命 $X$(单位:小时),则其可能的取值为 $[ 0,~ +\infty )$,
    这是一个无限总体的问题。
    进一步假设产品的寿命 $X$ 服从指数分布,则其概率密度函数为
    $$ f_X (x;~ \lambda) = \lambda {\rm e}^{-\lambda x} \cdot {\rm
      I}_{[ 0,~ +\infty )} (x) = \left\{
    \begin{array}{ll}  \lambda {\rm e}^{-\lambda x}, & x\geq 0 \\ 0,
      & x < 0 \end{array} \right. $$ }

    \begin{remark}
      示性函数
      \begin{equation}
       I_A(x) = \begin{cases}
         1& x\in A \\
         0& \text{其它}
       \end{cases}
      \end{equation}
    \end{remark}

  \item 一个总体有 $N$ 个元素,其指标分别为 $a_1 > a_2 > \cdots > a_N$,指定
  自然数 $M < N$,$n<N$,并设 $m=\frac{nM}{N}$ 为整数.~
  在 $\left( a_1,~a_2,~\ldots,~a_M \right)$ 中不放回地随机抽出 $m$ 个,在
  $\left( a_{M+1},~a_{M+2},~\ldots,~a_N \right)$
  中不放回地随机抽出 $n-m$ 个.~ 写出所得样本的分布. ~{\color{cyan} (2 分)}

  \begin{remark}
    从包含有 $n$ 个不同的元素的总体中取出 $r$ 个来进行排列,既要考虑到取出的元素
    也要顾及取出顺序. 这种排列分为两类
    \begin{itemize}
      \item 有放回地选取. 从 $n$ 个不同的元素中取出 $r$ 个元素进行排列,这种排
      列称为有重复的排列, 其总数共有 $n^r$ 种.
      \item 无放回地选取.
      从 $n$ 个不同的元素中取出 $r$ 个元素进行排列,其总数为
      \begin{equation}
       A_n^r = n(n-1)(n-2)\cdots(n-r+1)
      \end{equation}
     这种排列叫选排列,当 $r = n$ 时,称为全排列.
     \item $n$ 个不同的元素的全排列数为
     \begin{equation}
      P_n = n(n-1)(n-2)\cdots 3 \cdot 2 \cdot 1 = n\,!
     \end{equation}

    \end{itemize}
  \end{remark}

  {\color{red} \heiti 【解】} {\color{teal} \kaishu
    假设所抽取的样本为 $\left( X_1,~ X_2,~\ldots,~X_n\right)$,则前面的
    $m$ 个个体 $\left( X_1,~ X_2,~\ldots,~X_m\right)$ 是在
    $\left( a_1,~ a_2,~\ldots,~a_M\right)$ 中不放回地抽取,共有
    $$ {\rm A}_M^m = \binom{M}{m}= \displaystyle{\frac{M!}{m!}} $$
    种等可能的结果.~后面 $n-m$ 个个体 $\left( X_{m+1},~
    X_{m+2},~\ldots,~X_n \right)$ 是在
    $\left( a_{M+1},~ a_{M+2},~\ldots,~a_N \right)$ 中不放回地抽取,~共有
    $$ {\rm A}_{N-M}^{n-m} = \binom{N-M}{N-M} =
    \displaystyle{\frac{(N-M)!}{(n-m)!}} $$
    种等可能的结果.~故 $\left( X_1,~ X_2,~\ldots,~X_n \right)$ 共有
    $$ {\rm A}_M^m \cdot {\rm A}_{N-M}^{n-m} = \frac{M!}{m!} \cdot
    \frac{(N-M)!}{(n-m)!} $$
    种等可能的结果.~于是
    \begin{eqnarray*}
      P \left( X_1=x_1,~ \ldots,~ X_m=x_m,~ X_{m+1}=x_{m+1},~
      \ldots,~ X_n=x_n \right) &=&
      \frac{1}{{\rm A}_M^m \cdot {\rm A}_{N-M}^{n-m}} \\ && \\
      &=& \frac{m!(n-m)!}{M!(N-M)!}
    \end{eqnarray*}
  }

  \item 一物体的重量 $a$ 未知,有两架天平可用,其随机误差分别服从正态分布
  $N\left( 0,~\sigma^2_1 \right)$ 和$ N\left( 0,~\sigma^2_2 \right)$,
  其中 $\sigma^2_1$ 和 $\sigma^2_2$ 都未知.~先把物体在第一架天平上称两次得
  $X_1,~X_2$,
  再在第二架天平上称两次得 $X_3,~X_4$,然后视 $\left| X_1 - X_2 \right|
  \leq \left| X_3 - X_4 \right|$ 与否而在第一架或第二架天平上
  再称 $n-4$ 次得 $X_5,~\ldots,~X_n$.~写出 $\left(
  X_1,~X_2,~\ldots,~X_n \right)$ 的密度.~{\color{cyan} (2 分)}

  {\color{red} \heiti 【解】} {\color{teal} \kaishu
    由题意可知 $X_1,~ X_2 \sim N \left( a,~ \sigma^2_1 \right)$,~ 该总体
    的概率密度函数为
    $$ f_X (x) = \frac{1}{\sqrt{2\pi} \sigma_1} {\rm
      e}^{-\frac{(x-a)^2}{2\sigma^2_1}} $$
    同样,$X_3,~ X_4 \sim N \left( a,~ \sigma^2_2 \right)$,~ 该总体的概率
    密度函数为
    $$ f_X (x) = \frac{1}{\sqrt{2\pi} \sigma_2} {\rm
      e}^{-\frac{(x-a)^2}{2\sigma^2_2}} $$
    当 $\left| X_1 - X_2 \right| \leq \left| X_3 - X_4 \right|$  时,~
    $X_i \sim N \left( a,~ \sigma^2_1 \right),~ i=5,~6,~\ldots,~n$,~
    因此
    $$ f_{X_i} (x_i) = \frac{1}{\sqrt{2\pi} \sigma_1} {\rm
      e}^{-\frac{\left( x_i-a \right)^2}{2\sigma^2_1}} ,~
    i=5,~6,~\ldots,~n $$
    当 $\left| X_1 - X_2 \right| > \left| X_3 - X_4 \right|$  时,~
    $X_i \sim N \left( a,~ \sigma^2_2 \right),~ i=5,~6,~\ldots,~n$,~
    因此
    $$ f_{X_i} (x_i) = \frac{1}{\sqrt{2\pi} \sigma_2} {\rm
      e}^{-\frac{\left( x_i-a \right)^2}{2\sigma^2_2}} ,~
    i=5,~6,~\ldots,~n $$
    再根据简单随机样本的定义,我们有 $\left( X_1,~ X_2,~ X_3,~ X_4,~ X_5,~
    X_6,~ \ldots,~ X_n \right)$ 相互独立,于是其联合概率密度函数为

    $$ f_{X_1, X_2, \ldots, X_n} \left( x_1, x_2, \ldots, x_n \right)
    = f_{X_1} \left( x_1 \right) f_{X_2} \left( x_2 \right)
    f_{X_3} \left( x_3 \right) f_{X_4} \left( x_4 \right) f_{X_5}
    \left( x_5 \right) \cdots f_{X_n} \left( x_n \right) $$
    $$ = \left\{ \begin{array}{l}
      \displaystyle{\frac{1}{\left( \sqrt{2\pi} \right)^n
          \sigma_1^{n-2}\sigma_2^2}} \exp
      \left[ -\frac{\underset{i=1}{\overset{2}{\sum}} \left( x_i - a
        \right)^2 + \underset{j=5}{\overset{n}{\sum}}
        \left( x_j - a \right)^2}{2\sigma_1^2} -
      \frac{\underset{k=3}{\overset{4}{\sum}} \left( x_k - a
        \right)^2}{2\sigma_2^2} \right], \\
      \hspace{10cm} \left| X_1 - X_2 \right| \leq \left| X_3 - X_4
      \right| \\ \\
      \displaystyle{\frac{1}{\left( \sqrt{2\pi} \right)^n
          \sigma_1^{2}\sigma_2^{n-2}}} \exp
      \left[ -\frac{\underset{i=1}{\overset{2}{\sum}} \left( x_i - a
        \right)^2}{2\sigma_1^2} -
      \frac{\underset{j=3}{\overset{n}{\sum}} \left( x_j - a
        \right)^2}{2\sigma_2^2} \right], \\
      \hspace{10cm} \left| X_1 - X_2 \right| > \left| X_3 - X_4
      \right|
    \end{array} \right. $$
  }

  \item 设总体 $X$ 服从两点分布 $b(1,~p)$ (即 $P(X=1)=p$,$P(X=0)=1-p$),
  其中 $p$ 是未知参数,
  $\boldsymbol{X} = \left( X_1,~X_2,~X_3,~X_4,~X_5 \right)$ 为从此总体中
  抽取的简单样本,
  \begin{enumerate}
    \item 写出样本空间 $\mathscr{X}$ 和 $\boldsymbol{X}$ 的概率分布.
    ~{\color{cyan} (2 分)}

    {\color{red} \heiti 【解】} {\color{teal} \kaishu
      样本空间为
      $$ \mathscr{X} = \left\{ \left( X_1,~ X_2,~ X_3,~ X_4,~
      X_5\right) ~:~X_i = 0 ~ \text{或} ~1,~ i=1,~2,~3,~4,~5 \right\}
      $$
      $\boldsymbol{X}$ 的概率分布为
      $$ P \left( X_1=x_1,~X_2=x_2,~X_3=x_3,~X_4=x_4,~X_5=x_5 \right)
      = p^{\underset{i=1}{\overset{5}{\sum}} x_i}
      \left( 1 - p \right)^{5 - \underset{i=1}{\overset{5}{\sum}}
        x_i} $$
      其中 $x_i = 0 ~ \text{或} ~1,~ i=1,~2,~3,~4,~5 $. 或者
      $\underset{i=1}{\overset{5}{\sum}} X_i \sim b \left( 5,~ p
      \right)$.
    }

    \item 指出 $X_1+X_2$,$\underset{1\leq i \leq 5}{\min}
    X_i$,$X_5+2p$,$X_5-E\left(X_1\right)$,
    $\displaystyle{\frac{\left(
        X_5-X_1\right)^2}{D\left(X_1\right)}}$ 哪些是统计量,哪些不是统计量,并
    说明理由.~{\color{cyan} (2 分)}

    {\color{red} \heiti 【解】} {\color{teal} \kaishu
      因为 $X_1+X_2$ 和 $\underset{1\leq i \leq 5}{\min} X_i$ 是样本
      $\boldsymbol{X} = \left( X_1,~X_2,~X_3,~X_4,~X_5 \right)$
      的函数,
      且不含未知参数,因此它们是统计量.~
      $X_5+2p$,$X_5-E\left(X_1\right)$ 和 $\displaystyle{\frac{\left(
          X_5-X_1\right)^2}{D\left(X_1\right)}}$  虽然也是样本
      $\boldsymbol{X} = \left( X_1,~X_2,~X_3,~X_4,~X_5 \right)$
      的函数,
      但其中含有未知参数 $p$,所以它们不是统计量.
    }
  \end{enumerate}

  \item 设 $a\not= 0$ 和 $b$ 皆为常数,令
  $y_i=ax_i+b$,$i=1,~2,~\ldots,~n$.
  \begin{enumerate}
    \item 证明 $y_1,~y_2,~\ldots,~y_n$ 的样本均值 $\overline{y}$ 与
    $x_1,~x_2,~\ldots,~x_n$  的样本均值 $\overline{x}$ 之间的关系为
    $\overline{y} = a \overline{x} + b$.~{\color{cyan} (2 分)}

    {\color{red} \heiti 【证明】} {\color{teal} \kaishu
      证明过程如下:
      \begin{align*}
        \overline{y} &= \frac{1}{n} \sum^n_{i=1} y_i  \\
        &= \frac{1}{n} \sum^n_{i=1} \left( a x_i + b \right) \\
        &= a \left( \frac{1}{n} \sum^n_{i=1} x_i \right) +
        \frac{1}{n} \sum^n_{i=1} b \\
        &= a \overline{x} + b
      \end{align*}
    }

    \item 证明 $y_1,~y_2,~\ldots,~y_n$ 的样本方差 $S^2_y$ 与
    $x_1,~x_2,~\ldots,~x_n$  的样本方差 $S^2_x$ 之间的关系为
    $S^2_y = a^2S^2_x$.~{\color{cyan} (2 分)}

    {\color{red} \heiti 【证明】} {\color{teal} \kaishu
      证明过程如下:
      \begin{align*}
        S^2_y &= \frac{1}{n-1} \sum^n_{i=1} \left( y_i - \overline{y}
        \right)^2 \\
        &= \frac{1}{n-1} \sum^n_{i=1} \left[ \left( a x_i + b \right)
        - \left( a \overline{x} + b \right) \right]^2 \\
        &= \frac{a^2}{n-1} \sum^n_{i=1} \left( x_i - \overline{x}
        \right)^2 \\
        &= a^2 S^2_x
      \end{align*}
    }

    \item
    根据上述结果,利用适当的变换,求下列数据的样本均值和样本方差:~{\color{cyan}
      (2 分)}
    $$ 480, \quad 550, \quad 500, \quad 590, \quad 510, \quad 560,
    \quad 490, \quad 600, \quad 580. $$

    {\color{red} \heiti 【解】} {\color{teal} \kaishu
      做变换 $y = 10 x + 550$,则当 $x$ 取值
      $$ -7, \quad 0, \quad -5, \quad 4, \quad -4, \quad 1, \quad -6,
      \quad 5, \quad 3 $$
      时,$y$ 取值即为
      $$ 480, \quad 550, \quad 500, \quad 590, \quad 510, \quad 560,
      \quad 490, \quad 600, \quad 580. $$
      容易求得
      $$ \overline{x} = \frac{1}{9} \sum_{i=1}^9 x_i = -1 , \quad
      s^2_x = \frac{1}{9-1} \sum_{i=1}^9 \left( x_i - \overline{x}
      \right)^2 = 21 $$
      根据上述公式,我们就有
      $$ \overline{y} = 10 \overline{x} + 550 = 10 \times (-1) + 550
      = 540 , \quad
      s^2_y = 10^2 s^2_x = 2100 $$
    }
  \end{enumerate}
\end{enumerate}



\end{document}